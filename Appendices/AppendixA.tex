\chap{ Dotted or Undotted}

This appendix it's aimed to set up the conventions / notation that I will use in the rest of the thesis and to refresh the reader (and myself) some topics.
The metric is the mostly plus $ \eta = (-+++)$. The notation will be the same as the book  Wess \& Bagger [referecia]. If the reader is not familiar with these concepts keep going that in the end I will make a connection with the usual Dirac stuff.

Let $\mathbf{M}$ be a two-by-two matrix with $\det \mathbf{M}=1 $, i.e. $\mathbf{M} \in SL(2,C)$, this are matrices with complex values and unit determinant. One thing to note, is the number of generators of this group. We have $4$ complex entries ($8$ real) and the constrain from the unit determinant give two equations (real part $=1$ and  imaginary $=0$). Thus we have $8-2 = 6 $ generators, the same as our old friend The Lorentz Group $SO(3,1)$ with $3$ boosts $+\quad 3$ rotations. 
Now we introduce the the dotted and undotted indices. The spinor with dotted indices transform under the $(0,1/2)$ representation of Lorentz group and spinor with undotted indices transform  under $(1/2,0)$ conjugate representation.The spinor indices take values $\alpha = 1,2 \quad \dot{\alpha} = \dot{1},\dot{2}$.


\begin{subequations}
\begin{equation} 
\psi^{'}_{\alpha} = M_{\alpha}^{\;\;\beta} \psi_{\beta} \quad \quad; \quad \quad \psi^{'\alpha} =(M^{-1})_{\beta}^{\;\;\alpha}\psi^{\beta}  
\label{eq:1a}
\end{equation}

\noindent
%\text{and}

\begin{equation}
\bar{\psi}_{\dot{\alpha}}^{'} = (M^{*})_{\dot{\alpha}}^{\;\;\dot{\beta}}\bar{\psi}_{\dot{\beta}}\quad \quad; \quad \quad
\bar{\psi}^{'\dot{\alpha}} = (M^{*})_{\,\,\,\,\,\,\dot{\beta}}^{-1\,\,\,\dot{\alpha}}\bar{\psi}^{\dot{\beta}} 
\label{eq:1b}
\end{equation}
\end{subequations}


We have two indices (dotted and undotted) the two representation are inequivalent, i.e. we can not find a matrix $\mathbf{C}$ such that $\mathbf{M} = \mathbf{C} \mathbf{M}^{*} \mathbf{C}^{-1}  $. But the two representation in \eqref{eq:1a} are equivalent, thus exist a matrix $\mathbf{\varepsilon}$ such that $\mathbf{M} = \mathbf{\varepsilon} \mathbf{M}^{-1 T} \mathbf{\varepsilon}^{-1}$. Hang on that we will see what this matrix is. The same is valid for the two transformation with dotted indices.


We recall that any $2\times2$ matrix can be written as linear combination of the Pauli matrices plus the identity. Let me call this basis as $\sigma ^{m} = (-I,\vec{\sigma})$, where $m = 0,...,3$.


\begin{equation}
\mathbf{P} = P_{m} \mathbf{ \sigma^{m} } = -IP_{0} + \vec{P} \cdot \vec{\sigma} =
\begin{pmatrix} 
-P_{0} + P_{3} & P_{1} - iP_{2}\\
P_{1} + iP_{2} &-P_{0} - P_{3}  
\end{pmatrix}
\end{equation}


We can see that $\mathbf{P}$ is hermitian ($\mathbf{P} = \mathbf{P}^{\dagger}$). A nice property of the matrix $P$ is that $\det \mathbf{P}= P_{0}^{2} - \vec{P} \cdot \vec{P} = -\eta^{mn} P_{m} P_{n} $. Using the fact that $\mathbf{P}$ is hermitian we can write another matrix  $\mathbf{P'}$ as:

\begin{align} \label{eq:2}
\mathbf{P'}&= \mathbf{M}\mathbf{P}\mathbf{M}^{\dagger}\\
\mathbf{P'}^{\dagger}&= (\mathbf{M}\mathbf{P}\mathbf{M}^{\dagger})^{\dagger} = \mathbf{M}\mathbf{P}\mathbf{M}^{\dagger} = \mathbf{P'} 
\end{align}

Both $\mathbf{P'}$  and $\mathbf{P}$ can be written as linear combination of  $\sigma ^{m} $. The determinant of  $\mathbf{P'}$ (because the determinant of $\mathbf{M}$ is one and $\det [ABC] = \det [A]\det [B]\det [C]  $)  is equal to the determinant of $\mathbf{P'}$. 


\begin{equation}
\det \mathbf{P'} = -\eta^{mn} P_{m}^{'} P_{n}^{'}  = -\eta^{mn} P_{m} P_{n}
\end{equation}


 Now we start to see the connection between the Lorentz group and this matrices.This transformation correspond to a Lorentz transformation, that's cool. Before we continue let's appreciate what we have done. We stared defining a  $2\times2$  matrix $\mathbf{M}$ that had determinant one (you could say unimodular), and we noted that any $2\times2$ hermitian matrix $\mathbf{P}$  could be expanded as a linear combination of  $\sigma ^{m} $ and the determinant of this was the inner product of a Lorentz four vector, i.e, $\eta^{mn} P_{m} P_{n} $. Finally we found a transformation that is the same as the Lorentz Transformation.
 
 
 
 Lets take a look on the index structure of  $\mathbf{P}$. From \eqref{eq:1a} that  $\mathbf{M}^{\dagger} \equiv \mathbf{({M}^{T})^{*}} = ((M_{\alpha}^{\,\,\,\,\beta})^{T})^{*}  = (M_{\,\,\,\,\alpha}^{\beta})^{*}  =M_{\,\,\,\,\dot{\alpha}}^{\dot{\beta}} $. Thus we can rewrite \eqref{eq:2} as:
 
 \begin{equation}
  P_{\alpha \dot{\alpha}}^{'} =M_{\alpha}^{\,\,\,\,\beta} P_{\beta \dot{\beta}} M_{\,\,\,\,\dot{\alpha}}^{\dot{\beta}}
 \end{equation}
 
And the index structure of the Pauli matrices :  $\sigma ^{m} = \sigma ^{m}_{\alpha \dot{\alpha}}$. Note that we use Latin indices for vectors and tensors and Greek indices for spinors.
Now we return to the matrix $\mathbf{\varepsilon}$ that relate the two equivalent representation \eqref{eq:1a}.


Let


\begin{align}
\varepsilon = (\varepsilon_{\alpha \beta})=
\begin{pmatrix} 
\varepsilon_{11} &\varepsilon_{12}\\
\varepsilon_{21} & \varepsilon_{22} 
\end{pmatrix} =
\begin{pmatrix} 
0 &-1\\
1 & 0 
\end{pmatrix}\\
\varepsilon^{-1} = (\varepsilon^{\alpha \beta})=
\begin{pmatrix} 
\varepsilon^{11} &\varepsilon^{12}\\
\varepsilon^{21} & \varepsilon^{22} 
\end{pmatrix} =
\begin{pmatrix} 
0 &1\\
-1 & 0 
\end{pmatrix} 
\end{align}

with this matrix one can just plug in $\mathbf{M}^{-1T} = \mathbf{\varepsilon} \mathbf{M}^{} \mathbf{\varepsilon}^{-1}$ and check that works. The $\varepsilon_{\alpha \beta}$ and  $\varepsilon^{\alpha \beta}$ are antisymmetric tensors, and satisfy $\det \varepsilon = 1$ and $\varepsilon_{\alpha \beta} \varepsilon^{\beta \gamma} = \delta^{ \;\;\gamma}_{\alpha} $. 

we can write  $\mathbf{M}^{-1T} = \mathbf{\varepsilon} \mathbf{M}^{} \mathbf{\varepsilon}^{-1}$ with indices :

\begin{equation}
 \varepsilon^{ \alpha \beta} M^{\;\;\gamma}_{\beta} \varepsilon_{\gamma \rho} = (M^{-1T})^{\;\;\alpha}_{\rho} = (M^{-1})^{\alpha}_{\;\;\rho} 
\end{equation}


\begin{align}
\psi^{'\alpha} &= (M^{-1})^{\alpha}_{\;\;\rho} \psi^{\rho}  = \varepsilon^{ \alpha \beta} M^{\;\;\gamma}_{\beta} (\varepsilon_{\gamma \rho}\psi^{\rho} )
\\  
(\varepsilon_{ \beta \alpha}\psi^{'\alpha} ) &=   M^{\;\;\gamma}_{\beta} (\varepsilon_{\gamma \rho}\psi^{\rho} )
\end{align}

Thus $\varepsilon_{\gamma \rho}\psi^{\rho}$ transform as $\psi_{\gamma}$ and the $\varepsilon$ tensor can be used to lower and raise indices:
\begin{equation}
\psi_{\gamma}  = \varepsilon_{\gamma \rho}\psi^{\rho} \quad ; \quad \psi^{\gamma}  = \varepsilon^{\gamma \rho}\psi_{\rho}
\end{equation}


Every thing that we have done for undotted indice can be done similar for dotted.

We are almost ready to see the connection to Dirac usual spinor. Before to do we take a look on some identities of the Pauli matrix. If we define another Pauli basis:
\begin{subequations}
\begin{equation}
\bar{\sigma} ^{m} = (-I,-\vec{\sigma})
\end{equation}
 \text{with indices}
\begin{equation}
 (\bar{\sigma} ^{m})^{\dot{\alpha} \alpha} = \varepsilon^{\dot{\alpha}\dot{\beta}}\varepsilon^{\alpha\beta} (\sigma^{m})_{\beta \dot{\beta}}
\end{equation}
\end{subequations}

we have some important identities:

\begin{subequations}
\begin{equation}
 \sigma^{(m}\bar{\sigma}^{n)} = (\sigma^{m}\bar{\sigma}^{n} + \sigma^{n}\bar{\sigma}^{m})^{\;\;\beta}_{\alpha} = -2\eta^{mn}\delta^{\;\;\beta}_{\alpha}
\end{equation}
\begin{equation}
 \bar{\sigma}^{(m}\sigma^{n)} = (\bar{\sigma}^{m}\sigma^{n} + \bar{\sigma}^{n}\sigma^{m})^{\;\;\dot{\beta}}_{\dot{\alpha}} = -2\eta^{mn}\delta^{\;\;\dot{\beta}}_{\dot{\alpha}}
\end{equation}
\end{subequations}

and 

\begin{subequations}
\begin{equation}
 \text{Tr} \;\sigma^{m}\bar{\sigma}^{n}  = -2\eta^{mn}
 \label{Trace}
\end{equation}
\begin{equation}
(\sigma^{m})_{\alpha\dot{\alpha}} (\bar{\sigma}_{m})^{\dot{\beta}\beta}= -2 \delta^{\;\;\dot{\beta}}_{\dot{\alpha}}\delta^{\;\;\beta}_{\alpha}
\end{equation}
\end{subequations}




Now we can easy go back and forth between Lorentz indices and bispinor indices ($ m \leftrightarrow \alpha \dot{\alpha}$ ):


\begin{equation}
p_{\alpha \dot{\alpha}} = p_{m}(\sigma^{m})_{\alpha\dot{\alpha}} \quad ; \quad p_{m} =- \frac{1}{2} (\bar{\sigma}_{m})^{\dot{\beta}\beta} p_{\beta \dot{\beta}}
\end{equation}



As I promise, let's see the connection with the usual Dirac matrices and spinors. The Clifford algebra is (in the $(-,+++)$ metric):

\begin{equation}
\{\Gamma^{m} ,\Gamma^{n}\} = -2I\eta^{mn}
\label{cli}
\end{equation}

In the Weyl basis the gamma matrix is :

\begin{equation}
\Gamma^{m} =
\begin{pmatrix} 
0 & \sigma^{m}\\
\bar{\sigma}^{m} & 0 
\end{pmatrix}
\end{equation}

One can easily see that this gamma matrix satisfy the Clifford algebra \eqref{cli}. The gamma matrix act on a 4 components spinor with index structure: 

\begin{equation*}
\Psi =
\begin{pmatrix} 
 \psi_{\alpha}\\
\bar{\chi}^{\dot{\alpha}}   
\end{pmatrix}
\end{equation*}

One can write the Dirac equation:
\begin{subequations}
\begin{equation}
( \Gamma^{m}\partial_{m} + m )\Psi = 0
\end{equation}
\text{and in the weyl basis}
\begin{equation}
( (\bar{\sigma}^{m})^{ \dot{\alpha} \alpha}\partial_{m} + m )\psi_{\alpha} = 0
\end{equation}
\begin{equation}
((\sigma^{m})_{\alpha \dot{\alpha}} \partial_{m} + m )\bar{\chi}^{\dot{\alpha}} =0
\end{equation}
\end{subequations}


Remember also that the Lorentz generators were given by $S^{mn} = \frac{1}{4}[ \Gamma^{m}, \Gamma^{n}]$, and the Dirac component spinor transform as $\Psi \rightarrow \exp(\frac{1}{2}\omega_{mn}S^{mn})\Psi$. Then the Lorentz group generator in the spinor representation become:

\begin{subequations}
\begin{equation}
(\sigma^{mn})^{\;\;\beta}_{\alpha} =\frac{1}{4} (\sigma^{m}\bar{\sigma}^{n} - \sigma^{n}\bar{\sigma}^{m})^{\;\;\beta}_{\alpha}
\end{equation}
\begin{equation}
(\bar{\sigma}^{mn})^{\dot{\alpha}}_{\;\;\dot{\beta}} = \frac{1}{4}(\bar{\sigma}^{m}\sigma^{n} - \bar{\sigma}^{n}\sigma^{m})^{\dot{\alpha}}_{\;\;\dot{\beta}}
\end{equation}
\end{subequations}


This matrices seem strange when one looks for the first time, remember they are made of Pauli Matrices, in particular:
\begin{equation*}
\sigma^{ij} = \bar{\sigma}^{ij} = -\frac{i}{2}\epsilon^{ijk}\sigma^{k} \quad \text{and} \quad
\sigma^{0i} = -\bar{\sigma}^{0i} =\frac{1}{2}\sigma^{i}
\end{equation*}

these are rotations and boost respectively. The $\epsilon^{ijk}$ is the usual Levi-Civita symbol, and $i,j,k = 1,2,3$ with $\epsilon^{123} = 1$.  Note that rotations act the same in booth spinors as opposed for boost. The Lorentz transformation acts as:


\begin{equation}
\psi^{'}_{\alpha} = (e^{\frac{1}{2}\omega_{mn}\sigma^{mn}})_{\alpha}^{\;\;\beta} \psi_{\beta}
\label{eq:4}
\end{equation}
\begin{equation}
\bar{\psi}^{'\dot{\alpha}} = (e^{\frac{1}{2}\omega_{mn}\bar{\sigma}^{mn}})_{\;\;\dot{\beta}}^{\dot{\alpha}}\bar{\psi}^{\dot{\beta}} 
\end{equation}



Now let's make some checks and see that with these Lorentz transformation we get the right answers. We know what a rotation does on a vector $P^{m}$. Then if we use \eqref{eq:2} and multiply by $\bar{\sigma}^{m}$ and take the trace using \eqref{Trace} we get:


\begin{equation}
P^{'m} = -\frac{1}{2} \; \text{Tr} [\bar{\sigma}^{m}\mathbf{M}\sigma^{n}\mathbf{M^{\dagger}}]P_{n}
\label{eq5}
\end{equation}

If we choose a rotation on z-axis:

\begin{equation*}
\mathbf{M} = e^{\omega_{12}\sigma^{12}} = e^{\frac{i}{2}\theta\sigma^{3}} = I\cos(\theta/2) + i\sigma^{3}\sin(\theta/2)
\end{equation*}

where we used the fact $\omega_{12} = -\omega_{21} $ that kills the $1/2$ in \eqref{eq:4} and then we choose  $\omega_{12} = - \theta$. Now that we have all the elements we can express \eqref{eq5} as:

\begin{equation}
P^{'m} = P^{m} \cos^{2}(\theta/2)  -\frac{1}{2}P_{n} \sin^{2}(\theta/2) \text{Tr} [\bar{\sigma}^{m}\sigma^{3}\sigma^{n}\sigma^{3}]   -\frac{i}{2}\sin(\theta/2)\cos(\theta/2)P_{n}\text{Tr} [\bar{\sigma}^{m}[\sigma^{3},\sigma^{n}]]
\end{equation}



with this expression and plus some sigma Trace identities and trigonometric one find:



\begin{equation}
P^{'0} = P^{0} \quad ; \quad \vec{P}^{'} = \begin{pmatrix} 
\cos(\theta) & \sin(\theta) & 0\\
-\sin(\theta) & \cos(\theta) & 0\\
0 &	0	&	1
\end{pmatrix} \vec{P} 
\end{equation}


That's our rotation matrix, it worked! Let me say the trivial fact that if $\theta = 2\pi$ then $\vec{P}^{'}=\vec{P}$. Now we do the same trivial statement for a spinor, under a full rotation $\psi^{'} = e^{\frac{i}{2} 2\pi\sigma^{3}}\psi = -\psi$, that's another way to see that it's a (1/2) 







\begin{equation}
a \, \overset{\eqref{eq:1a}}{=} \, b
\end{equation}